%________________________________________________________________________________________
% @brief    LaTeX2e Resume for Christopher A. Hamm
% @author   Chris Hamm
% @date    2017-04-24
% @info     Based on Latex Resume Template by Chris Paciorek 
%           http://www.biostat.harvard.edu/~paciorek/

%________________________________________________________________________________________
\documentclass[margin,line, a4paper, sans, 11pt]{resume}

\usepackage{hyperref}
\usepackage{color}
\usepackage{fontawesome}
\newcommand{\codechunk}[1]{\texttt{#1}}
\hypersetup{colorlinks = true, urlcolor = blue}
\pagenumbering{arabic}
\pagestyle{plain}

\begin{document}
\name{\Large Christopher A. Hamm}

\begin{resume}

%_______________________________________________________________________________________
{\small (As of 21 May 2017)}

\section{\mysidestyle Contact information}
37437 State Hwy. 16 Woodland, CA 95695 USA \\
+1 (530) 669-6107\\
\href{mailto:christopher.a.hamm@monsanto.com}{christopher.a.hamm@monsanto.com}\\
\faTwitter \quad \faTwitterSign \href{http://twitter.com/butterflyology}{\codechunk{butterflyology}}  \\
\faGithub \quad \faGithubSign \href{http://github.com/butterflyology}{\codechunk{butterflyology}}


\section{\mysidestyle Appointments}
 
{\bf Data Scientist}, 
Monsanto Company \\
Woodland, CA, 2017-Present.


\section{\mysidestyle Education}

{\bf University of California, Davis}; Brown Lab (postdoc)\\
	Population Health \& Reproduction; 2016-2017.
    
    
{\bf University of Kansas}; Walters Lab (postdoc)\\
	Ecology \& Evolutionary Biology; 2014-2016  
    

{\bf University of California, Davis}; Begun \& Turelli Labs (postdoc)\\
	Evolution \& Ecology; 2012-2014


{\bf Michigan State University}; Landis \& Williams Labs (graduate student)
	\begin{itemize}
	\vspace{2mm}
	
		\item	Entomology; PhD., 2012 
		\item Ecology, Evolutionary Biology \& Behavior; PhD., 2012
		
	\end{itemize}


{\bf California State University, Fresno}; Crosbie Lab (graduate student)
	\begin{itemize}
	\vspace{2mm}
	
		\item Biology; M.S., 2008 {\em with distinction}
		
	\end{itemize}


{\bf California State University, Fresno}
	\begin{itemize}
	\vspace{2mm}
		
		\item Biology; B.S., 2004 {\em Magna Cum Laude}
	\end{itemize}


\section{\mysidestyle Publications}

{\bf Hamm, C.A}. Chromosome Number of the Monarch Butterfly, {\em Danaus plexippus} (Linnaeus 1758) and the Danainae. \href{http://biorxiv.org/content/early/2017/02/08/107144}{BioR$\chi$iv}.

Marion, Z.H. and {\bf C.A. Hamm}. A hierarchical Bayesian approach to estimate endosymbiont infection rates. \href{http://biorxiv.org/content/early/2017/01/21/102020}{BioR$\chi$iv}.

{\bf Hamm, C.A.}, C.M. Penz and P.J. DeVries. Wing shape evolution in {\it Hamadryas} butterflies corresponds to vertical microhabitat use and species range size. {\em In revision}.

Bell, K.L., {\bf C.A. Hamm}, A.M. Shapiro and C.C. Nice. Sympatric, temporally isolated populations of the Pine White butterfly {\em Neophasia menapia}, are morphologically and genetically differentiated. In press, {\em PLoS ONE}.

{\bf Hamm, C.A.} and J.A. Fordyce. 2016. Greater host breadth still not associated with increased diversification rate in the Nymphalidae - a reply to Janz et al. \href{http://onlinelibrary.wiley.com/doi/10.1111/evo.12914/abstract}{\em Evolution} 70: 1156-1160.

{\bf Hamm C.A.} and J.A. Fordyce. Selaginella and the satyr: {\em Euptychia westwoodi} oviposition preference and performance. \href{http://jinsectscience.oxfordjournals.org/content/16/1/39}{\em Journal of Insect Science} 16: 39; 1-4.

\newpage

Fordyce J.A., C.C. Nice, {\bf C.A. Hamm} and M.L. Forister. Quantifying diet breadth through ordination of host associations. \href{http://onlinelibrary.wiley.com/doi/10.1890/15-0093.1/abstract}{\em Ecology} 97: 842-849.

DeVries, P.J., {\bf C.A. Hamm}, and J.A. Fordyce. 2016. \href{https://www.researchgate.net/publication/301283999_A_Standardized_Sampling_Protocol_for_Fruit-Feeding_Butterflies_Nymphalidae}{\em Fruit-feeding butterflies (Nymphalidae) - standardized butterfly sampling protocol}. In Larson, T.H. (ed.). Core Standardized Methods for Rapid Biological Assessment. Conservation International, Arlington, VA.

{\bf Hamm C.A.} and J.A. Fordyce. 2015. Patterns of diversification and host plant utilization in the brush footed butterflies. \href{http://onlinelibrary.wiley.com/doi/10.1111/evo.12593/suppinfo}{\em Evolution} 63: 589-601 [cover article].

{\bf Hamm C.A.}, D.J. Begun, A. Vo, C.C.R. Smith, P. Saelao, A.O. Shaver, J. Jaenike, and M. Turelli. 2014. {\em Wolbachia} do not live by reproductive manipulation alone: infection polymorphism in {\em Drosophila suzukii} and {\em D. subpulchrella}. \href{http://onlinelibrary.wiley.com/doi/10.1111/mec.12901/abstract}{\em Molecular Ecology} 23: 4871-4885.

Tochen S., D.T. Dalton, N. Wiman, {\bf C.A. Hamm}, P.W. Shearer and V.M. Walton. 2014. Temperature-related development and population parameters for {\em Drosophila suzukii} (Diptera: Drosophilidae) on cherry and blueberry. \href{https://www.researchgate.net/publication/260681845_Temperature-Related_Development_and_Population_Parameters_for_Drosophila_suzukii_(Diptera_Drosophilidae)_on_Cherry_and_Blueberry}{\em Environmental Entomology} 42: 501-510.

{\bf Hamm C.A.}, C.A. Handley, A. Pike, M.L. Forister, J.A. Fordyce and C.C. Nice. 2014. {\em Wolbachia} infection and Lepidoptera of conservation concern. \href{http://www.insectscience.org/14.6/}{\em Journal of Insect Science} 14:6.

{\bf Hamm C.A.}, V. Rademacher, D.A. Landis and B.L. Williams. 2013. Conservation genetics and the implication for recovery of the endangered Mitchell{\textsc{\char13}}s satyr butterfly. \href{https://www.researchgate.net/publication/258044323_Conservation_Genetics_and_the_Implication_for_Recovery_of_the_Endangered_Mitchell%27s_Satyr_Butterfly_Neonympha_mitchellii_mitchellii}{\em Journal of Heredity} 105: 19-27.

Chiu J.C., X. Jiang, L. Zhao, {\bf C.A. Hamm}, J.M. Cridland, P. Saelao, K.A. Hamby, E.K. Lee, R.S. Kwok, G. Zhang, F.G. Zalom, V.M. Walton and D. J. Begun. 2013. Genome of Drosophila suzukii, the Spotted Wing Drosophila. \href{http://www.g3journal.org/content/early/2013/10/14/g3.113.008185.abstract?sid=c3350324-a08e-41d3-844b-7028546da365}{\em G3: Genes | Genomes | Genetics} 3: 2257-2271.

{\bf Hamm C.A.} 2013. Estimating abundance of the federally endangered Mitchell{\textsc{\char13}}s satyr butterfly using hierarchical distance sampling. \href{https://www.researchgate.net/publication/236213552_Estimating_abundance_of_the_federally_endangered_Mitchell%27s_satyr_butterfly_using_hierarchical_distance_sampling}{\em Insect Conservation and Diversity} 6: 619-626.

{\bf Hamm C.A.}, B.L. Williams and D.A. Landis. 2013. Natural history and conservation status of the endangered Mitchell{\textsc{\char13}}s Satyr butterfly: an update and expansion of our knowledge regarding {\em Neonympha mitchellii mitchellii} French 1889. \href{https://www.researchgate.net/publication/236213551_Natural_history_and_conservation_status_of_the_endangered_Mitchell's_satyr_butterfly_synthesis_and_expansion_of_our_knowledge_regarding_Neonympha_mitchellii_mitchellii_French_1889?ev=prf_pub}{\em Journal of the Lepidopterist{\textsc{\char13}}s Society} 67: 15-28.

{\bf Hamm C.A.} 2012. What pollinates {\em Lantana camara} (Verbenaceae) in the mountains of Costa Rica?  \href{https://www.researchgate.net/publication/236213554_What_pollinates_Lantana_camara_in_the_mountains_of_Costa_Rica}{\em Journal of Tropical Ecology} 28: 313-315.

Yanoviak S.P., S. Silveri, {\bf C.A. Hamm} and M. Solis. 2012. Stem characteristics and ant body size in a Costa. \href{https://www.researchgate.net/publication/236213554_What_pollinates_Lantana_camara_in_the_mountains_of_Costa_Rica}{\em Journal of Tropical Ecology} 28: 199-204.

{\bf Hamm C.A.} 2012. Development of Polymorphic Anonymous nuclear DNA markers for the endangered Mitchell{\textsc{\char13}}s satyr butterfly, {\em Neonympha mitchellii mitchellii} (Lepidoptera: Nymphalidae). \href{https://www.researchgate.net/publication/236213556_Development_of_polymorphic_anonymous_nuclear_DNA_markers_for_the_endangered_Mitchells_satyr_butterfly_Neonympha_mitchellii_mitchellii_%28Lepidoptera_Nymphalidae%29}{\em Conservation Genetics Resources} 4: 127-128.

\newpage

Landis D.A., A.K. Fiedler, {\bf C.A. Hamm}, D.L. Cuthrell, E.H. Schools, D.R. Pearsall, M.E. Herbert and P.J. Doran. 2011. Insect conservation in Michigan prairie fen: addressing the challenge of global change. \href{https://www.researchgate.net/publication/225462491_Insect_conservation_in_Michigan_prairie_fen_addressing_the_challenge_of_global_change}{\em Journal of Insect Conservation} 16: 131-142

{\bf Hamm C.A.} 2010. Multivariate discrimination and description of a new species of {\em Tapinoma} from the Western United States. \href{https://www.researchgate.net/publication/232694914_Multivariate_Discrimination_and_Description_of_a_New_Species_of_Tapinoma_from_the_Western_United_States}{\em Annals of the Entomological Society of America} 103: 20-29.

{\bf Hamm C.A.}, D. Aggarwal and D.A. Landis. 2010. Evaluating the impact of non-lethal DNA sampling on two butterflies, {\em Vanessa cardui} and {\em Satyrodes eurydice}. \href{https://www.researchgate.net/publication/226132277_Evaluating_the_impact_of_non-lethal_DNA_sampling_on_two_butterflies_Vanessa_cardui_and_Satyrodes_eurydice}{\em Journal of Insect Conservation} 14: 11-18.

{\bf Hamm C.A.} and B. Kamansky. 2009. New record of {\em Messor chicoensis} from the San Joaquin Valley of California. \href{https://www.researchgate.net/publication/236213557_New_Record_of_Messor_chicoensis_%28Hymenoptera_Formicidae%29_from_the_San_Joaquin_Valley_of_California}{\em Sociobiology} 53 (2B): 543-547.

{\bf Hamm C.A.} 2008. Designation of a neotype for Mitchell{\textsc{\char13}}s satyr. \href{https://www.researchgate.net/publication/236213560_Designation_of_a_neotype_for_Mitchell%27s_satyr_Neonympha_mitchellii_%28Lepidoptera_Nymphalidae%29}{\em Great Lakes Entomologist} 40: 201-202.

\section{\mysidestyle Data \& Software}

Bell, K.L., {\bf C.A. Hamm}, A.M. Shapiro and C.C. Nice. Data and code from Sympatric, temporally isolated populations of the Pine White butterfly {\em Neophasia menapia}, are morphologically and genetically differentiated. \href{https://zenodo.org/record/401052}{\codechunk Zenodo DOI}.

{\bf Hamm, C.A.} \href{https://github.com/butterflyology/spaceMovie}{\codechunk spaceMovie}: a Star Wars color palette generator for {\codechunk R}. \href{https://zenodo.org/record/249921}{\codechunk Zenodo DOI}.

Marion, Z.H. and {\bf Hamm, C.A.} 2016. Release 1.0 of data and code from \href{http://biorxiv.org/content/early/2017/01/21/102020}{BioR$\chi$iv} pre-print. \href{http://doi.org/10.5281/zenodo.166803}{\codechunk Zenodo DOI}.

{\bf Hamm, C.A.} and J.A. Fordyce. 2016. Data and code from {\em Evolution} 70: 1156-1160. \href{http://datadryad.org/resource/doi:10.5061/dryad.3c7jb}{Data Dryad}.

{\bf Hamm, C.A.} and J.A. Fordyce. 2016. Data and code from {\em Journal of Insect Science} 16: 39; 1-4. \href{https://figshare.com/articles/Euptychia_oviposition_and_larval_performance/3083320/1}{FigShare}.

Fordyce J.A., C.C. Nice, {\bf C.A. Hamm} and M.L. Forister. 2016. \href{https://cran.r-project.org/web/packages/ordiBreadth/index.html}{\codechunk ordiBreadth}: an {\codechunk R} package to calculate Ordinated Diet Breadth.

{\bf Hamm, C.A.} and J.A. Fordyce. 2015. Data and code from {\em Evolution} 63: 589-601. \href{http://datadryad.org/resource/doi:10.5061/dryad.p5v8g}{Data Dryad}.

{\bf Hamm C.A.}, D.J. Begun, A. Vo, C.C.R. Smith, P. Saelao, A.O. Shaver, J. Jaenike, and M. Turelli. 2014. Data cand code from {\em Molecular Ecology} 23: 4871-4885. \href{http://datadryad.org/resource/doi:10.5061/dryad.0pg63}{Data Dryad}.

{\bf Hamm C.A.}, V. Rademacher, D.A. Landis and B.L. Williams. 2013. Data and code from {\em Journal of Heredity} 105: 19-27. \href{http://datadryad.org/resource/doi:10.5061/dryad.n31kq}{Data Dryad}

{\bf Hamm, C.A.} 2013. Data and code from {\em Insect Conservation and Diversity} 6: 619-626. \\
\href{http://datadryad.org/resource/doi:10.5061/dryad.v977b}{Data Dryad}.

{\bf Hamm, C.A.} 2010. Data and code from {\em Annals of the Entomological Society of America} 103: 20-29.
\href{http://figshare.com/articles/Tapinoma_sessile_and_T_schriberi/842699}{FigShare}.

{\bf Hamm C.A.}, D. Aggarwal and D.A. Landis. 2010. Data and code from {\em Journal of Insect Conservation} 14: 11-18. \href{http://figshare.com/articles/Wing_clip_data/842705}{FigShare}.

\newpage


\section{\mysidestyle Grants}

{\bf Loewy Family Foundation Fellowship} (\$10,000) Mohonk Preserve. Award to conduct standardized butterfly monitoring of fruit-feeding butterflies at the, 2016.

{\bf Scriber Scholars Award in Butterfly Conservation} (\$1,000) Michigan State University. Award to conduct line transect distance sampling of the Mitchell{\textsc{\char13}}s satyr butterfly, 2011/2012.

{\bf Great Lakes Restoration Initiative Program Endangered Species Grant} (\$215,000) United States Fish and Wild Service.  Award \#PSGP-07-11 to {\bf C.A. Hamm} (author) and D.A. Landis (PI), to study the population genetics and reproductive parasite status of {\em Neonympha mitchellii mitchellii} and the reproductive parasite status of {\em Somatochlora hineana}, 2010.

{\bf Graduate Research Enhancement Grant} (\$1,300) Michigan State University, 2010.

{\bf Post-Course Research Award} (\$670) Organization for Tropical Studies. Award to conduct research on butterfly pollination of {\em Lantana camera} and {\em L. trifolia} at the Las Cruces Biological Station, 2010.

{\bf Scriber Scholars Award in Butterfly Conservation} (\$1000) Michigan State University. Award to assay Lepidoptera of conservation concern for the bacterium {\em Wolbachia}, 2009/2010.

{\bf Preventing Extinction Funding Grant} (\$50,000) United States Fish and Wild Service. Award to {\bf C.A. Hamm} (author) and D.A. Landis (PI) to conduct research on the status of the bacterium {\em Wolbachia} in the Mitchell{\textsc{\char13}}s satyr butterfly, 2009.

{\bf Theodore Roosevelt Memorial Fund} (\$1,500) American Museum of Natural History. Award to investigate phylogenetic relationships in the butterfly genus {\em Neonympha}, 2009.

{\bf G.H. Lauff Research Scholar Award} (\$1,500) Michigan State University. Research grant to conduct genetic sampling of the Mitchell{\textsc{\char13}}s satyr butterfly, 2009.

{\bf Council of Graduate Students Conference Grant} (\$300) Michigan State University, 2009.

{\bf Graduate Research Enhancement Grant} (\$2,400) Michigan State University, 2008.


\section{\mysidestyle Invited Presentations}

\begin{itemize}
\item ``Why are there so many butterflies?'' Monsanto Company, December 2016.

\item ``Host breadth, host shifts, and herbivore diversification.'' XXV International Congress of Entomology "From diet breadth to diversification: understanding host shifts in phytophagous insects" symposium. September 2016.

\item ``Why are there so many butterflies?'' University of Texas at Tyler, June 2016.

\item ``Teaching computational skills to researchers.'' SEARCH Symposium ({\bf S}cientists {\bf E}xploring non-{\bf A}cademic ca{\bf R}eer {\bf CH}oices). University of Kansas, April 2016.

\item ``The Most Common Infection in the World.'' Nerd Night Lawrence , January 2016.

\item ``Detecting sex-linked dosage compensation using RNASeq.'' Next Generation Sequencing Summer Course 2015, Michigan State University, August, 2015.

\item ``The 150-million-year hangover: side-effects of sexual reproduction in the Lepidoptera.'' Texas State University. April, 2015.

\newpage

\item ``The 150-million-year hangover: side-effects of sexual reproduction in the Lepidoptera.'' University of Nebraska, Lincoln. February, 2015.

\item ``Butterflies, flies, and bacteria.'' University of Kansas. March, 2014.

\item ``What is a Mitchell{\textsc{\char13}}s satyr Butterfly? Species concepts and conservation genetics of the federally endangered Mitchell{\textsc{\char13}}s satyr butterfly, {\em Neonympha mitchellii mitchellii}.'' University of New Orleans. November 2011.

\item ``What is a Mitchell{\textsc{\char13}}s satyr Butterfly? Species concepts and conservation genetics of the federally endangered Mitchell{\textsc{\char13}}s satyr butterfly, {\em Neonympha mitchellii mitchellii}.'' Program in Ecology, Evolutionary Biology \& Behavior Graduate Colloquium, Michigan State University. April, 2011.

\item ``Conservation of the Mitchell{\textsc{\char13}}s satyr butterfly'' 2009 Annual Meeting of the Entomological Society of America. December, 2009.
\end{itemize}


\section{\mysidestyle Contributed Presentations}

\begin{itemize}

\item ``Assembling transcriptomes for {\em Heliconius melpomene}, {\em H. cydno}, and {\em H. erato}.'' University of Cambridge. October 2015.

\item ``What is the Mitchell{\textsc{\char13}}s satyr butterfly? Contemporary approaches to an old question.'' 7th International Conference on the Biology of Butterflies, University of Turku, Finland. August 2014.

\item ``The Mitchell{\textsc{\char13}}s satyr butterfly: what is it and how many are there?'' Mitchell{\textsc{\char13}}s satyr Recovery Team Annual meeting. March 2012.

\item ``Molecular Ecology of the Endangered Mitchell{\textsc{\char13}}s Satyr Butterfly.'' Colorado State University. February 2011.

\item ``Population Genetics of the Endangered Mitchell{\textsc{\char13}}s Satyr Butterfly.'' Annual Meeting of the Entomological Society of America. December 2010.

\item ``Population Genetics and Reproductive Parasites in the Mitchell{\textsc{\char13}}s Satyr Butterfly.'' Mitchell{\textsc{\char13}}s satyr Recovery Team Annual Meeting. March 2010.

\item ``Multivariate Discrimination and Description of a new species of {\em Tapinoma} from the Western United States.'' Annual Meeting of the Entomological Society of America. December 2009.

\item ``Disruption of historic metapopulation dynamics and the endangered Mitchell{\textsc{\char13}}s satyr butterfly.'' Conf�rence Universitaire de Suisse Occidental workshop ``Evolution in Metapopulations''. September 2009.

\item ``Update on the {\em Neonympha mitchellii} Genetics Study.'' Seminar. Mitchell{\textsc{\char13}}s satyr Recovery Team Annual Meeting. March 2009.

\item ``Genetics, Conservation, and Mitchell{\textsc{\char13}}s Satyr.'' Seminar. Mitchell{\textsc{\char13}}s satyr Recovery Team Annual Meeting. March 2008.

\item ``The Impact of Non-Lethal Sampling on Two Species of Butterfly.'' Stewardship Network Conference. January 2008.

\item ``The Impact of Non-Lethal Sampling on Two Species of Butterfly.'' Annual Meeting of the Entomological Society of America. December 2007.

\end{itemize}


\section{\mysidestyle Teaching experience}

{\bf Certified Instructor Trainer:} Software Carpentry Foundation, Lead instructor training workshops. 
\\Workshops taught: 
\begin{itemize}

\item Berkeley Institute for Data Science \& Lawrence Berkeley Laboratories, August 2017 [scheduled].

\end{itemize}

\newpage

{\bf Certified Instructor:} Software/Data Carpentry Foundation, an organization whose volunteer members teach researchers basic software and computational skills. 
\\Workshops taught: 
\begin{itemize}

\item Software Carpentry, \codechunk{R}, Federal Reserve Board of Washington D.C. June 2017. [scheduled].

\item Data Carpentry, Genomics, California State University Monterey Bay. March 2017.

\item Software Carpentry, \codechunk{R}, Federal Reserve Bank of Kansas City. February 2017.

\item Software Carpentry, \codechunk{R}, Purdue University. October 2016.

\item Software Carpentry, \codechunk{R}, Federal Reserve Board of Washington D.C. August 2016.

\item Data Carpentry, \codechunk{R} ecology, Stony Brook University. August 2016.

\item Software Carpentry, \codechunk{R}, Federal Reserve Board of Washington D.C. June 2016.

\item Software Carpentry, \codechunk{R}, University of Connecticut. March 2016.

\item Data Carpentry, \codechunk{R} Ecology, University of Notre Dame. March 2016.

\item Software Carpentry, \codechunk{R}, University of California San Francisco. November 2015.

\item Data Carpentry, Genomics, Kellogg Biological Station. August 2015.

\item Software Carpentry, \codechunk{R}, Michigan State University. August 2015.

\item Software Carpentry, \codechunk{Python}, National Center for Atmospheric Research. April 2015.

\end{itemize}

{\bf Data Intensive Biology Summer Institute} - Reproducible Research with {\codechunk R}, one-week workshop on best practices and  how to incorporate those tools into research. University of California, Davis. July 2017 [scheduled].

{\bf Next Generation Sequencing analysis workshop} One-week advanced workshop on the analysis of NGS data. Kellogg Biological Station, Michigan. August 2015.

{\bf Software Carpentry Instructor Training Workshop}. Required course to become a certified instructor for the Software Carpentry Foundation and Data Carpentry Foundation, non-profit organizations whose mission is to teach researchers basic software skills. University of California, Davis. January 2015.

{\bf Teaching Associate} California State University, Fresno. \\Laboratories taught:
\begin{itemize}

\item Biological Sciences 1B - Introductory Biology for Science Majors.
	\begin{itemize}

		\item Spring 2006
		\item Fall 2005
		\item Spring 2005
		\item Fall 2005
		\item Spring 2005								
		\item Fall 2004
	
	\end{itemize}
	
	
\item Biology 10 - Introductory Biology for non-science majors.
	\begin{itemize}

		\item Fall 2005		
		\item Spring 2005

	\end{itemize}
	
\item Zoology 10 - Introductory Zoology for non-science majors.
	\begin{itemize}
	
		\item Spring 2006	
	
	\end{itemize}
	
\end{itemize}

\newpage

\section{\mysidestyle Specialized Training}

{\bf Software Carpentry Instructor Trainer}. Certification procedure to train and certify instructors for Software Carpentry. Spring 2017 [in progress].

{\bf Data Carpentry Curriculum Development Meeting}. Three-day workshop to develop and assess core curriculum for the reproducible research with \codechunk{Python} and \codechunk{Jupyter Notebook}. Berkeley Institute for Data Science, California. January 2017.

{\bf Work with Data Institute}. One-week workshop focusing on working with geospatial and remote sensing data. {\bf N}ational {\bf E}arth {\bf O}bservation {\bf N}etwork. June 2016.

{\bf Intermediate Bioinformatics Workshop}. One week course on high-level genomic analysis. Bodega Marine Laboratory. February 2016.

{\bf Applied Bayesian Modeling in R}. One-week advanced workshop on the theory and application of Bayesian statistics. {\bf S}cottish {\bf C}entre for {\bf E}cology and the {\bf N}atural {\bf E}nvironment. October 2015.

{\bf Data Carpentry Curriculum Development Meeting}. Three-day workshop to develop and assess core curriculum for the genomics teaching module. Cold Spring Harbor Laboratory, New York. March 2015.

{\bf Genome Assembly Masterclass}. One week workshop on the finer points of genome and metagenome assembly. University of California, Davis. December 2013.

{\bf Analysis of Organismal Form}. Semester-long class on the methods of geometric morphometrics and examples of their application in various biological disciplines. Online course, University of Manchester. Fall, 2012.

{\bf Hierarchical models for abundance, distribution and species richness in spatially structured populations using unmarked/R and WinBUGS}. One-week workshop on ecological modeling. USGS Patuxent Wildlife Research Center. April 2012.

{\bf Introduction to Coalescent Theory}. One-week workshop hosted by the Population Genomics Program and the Doctoral Program in Ecology and Evolution. University of Bern, Switzerland. September 2011.

{\bf Analyzing Next-Generation Sequencing Data}. Two-week workshop hosted by the Department of Computer Science and Engineering. Michigan State University. June 2011.

{\bf Summer Institute in Statistical Genetics}. Specialty workshops hosted by the Department of Biostatistics at the University of Washington. Modules attended: Statistical Computing, Population Genetic Data Analysis, Bayesian Inference, and Markov Chain Monte Carlo Simulations. June 2010.

{\bf Tropical Butterfly Ecology}, Organization for Tropical Studies Graduate Course. An intensive, two-week field course in Costa Rica focusing on tropical butterfly morphology, ecology, systematics and behavior. May 2010.

{\bf Tropical Biology: An Ecological Approach}: Organization for Tropical Studies Graduate Course. An intensive, mobile, eight-week field course in Costa Rica that exposed students to tropical ecosystems, research design and practice, and the fundamentals of tropical field biology. Spring 2010.

\newpage

{\bf Evolution in Metapopulations workshop}. A three-day workshop examining recent advances in metapopulations modeling. Conf�rence Universitaire de Suisse Occidental. August 2009.

{\bf Recent Advances in Conservation Genetics Workshop}. Sponsored by the Laboratory of Genomic Diversity at the National Institutes of Health and the Smithsonian Topical Research Institute. January 2009.

{\bf Scientific Research Diver} NAUI certified Master SCUBA diver and certified by the American Association of Underwater Scientists as a Scientific Research Diver. Spring 2006.


\section{\mysidestyle External Reviewer}

\begin{center}
\begin{tabular}{ c c }
Biological Invasions & BMC Evolutionary Biology \\
BMC Genetics & The Canadian Entomologist \\
Ecological Entomology & Evolutionary Applications \\
Evolution & Florida Entomologist \\
Frontiers in Zoology & Gene \\
Heredity & Insect Conservation and Diversity \\
Insectes Sociaux & Journal of Insect Physiology \\
Journal of Insect Conservation & Myrmecological News\\
\end{tabular}
\end{center}


\end{resume}
\end{document}
